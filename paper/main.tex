\documentclass[10pt,twocolumn]{article}

% Self-contained arXiv-style formatting — paste into Overleaf and compile
\usepackage[top=1in, bottom=1in, left=0.75in, right=0.75in]{geometry}
\usepackage[T1]{fontenc}
\usepackage{times}
\usepackage{amsmath,amssymb}
\usepackage{booktabs}
\usepackage{graphicx}
\usepackage[hidelinks]{hyperref}
\usepackage{xcolor}
\usepackage{multirow}
\usepackage{subcaption}
\usepackage[numbers,sort&compress]{natbib}
\usepackage{microtype}
\usepackage{enumitem}
\usepackage{caption}
\usepackage{fancyhdr}
\usepackage{titlesec}
\usepackage{abstract}
\usepackage{algorithmic}
\usepackage{algorithm}
\usepackage{float}

% --- Typography ---
\captionsetup{font=small,labelfont=bf}
\setlength{\columnsep}{0.3in}
\setlength{\parskip}{0.0em}
\setlength{\parindent}{1em}

% --- Section formatting ---
\titleformat{\section}{\large\bfseries}{\thesection}{1em}{}
\titleformat{\subsection}{\normalsize\bfseries}{\thesubsection}{0.8em}{}
\titlespacing*{\section}{0pt}{1.2ex plus 0.4ex minus 0.2ex}{0.8ex plus 0.2ex}
\titlespacing*{\subsection}{0pt}{1.0ex plus 0.3ex minus 0.1ex}{0.6ex plus 0.1ex}

% --- Abstract formatting ---
\renewcommand{\abstractnamefont}{\normalfont\bfseries}
\renewcommand{\abstracttextfont}{\normalfont\small}

% --- Headers ---
\pagestyle{fancy}
\fancyhf{}
\renewcommand{\headrulewidth}{0pt}
\fancyfoot[C]{\thepage}

\title{Equitas: A Benchmark for Corruption-Robust Aggregation\\in Hierarchical Multi-LLM Committees}

\author{
  Akshan Krithick \\
  University of Washington \\
  \href{mailto:akshankrithick305@gmail.com}{\texttt{akshan3@uw.edu}}
}

\date{}

\begin{document}

\maketitle

% ===========================================================================
\begin{abstract}
% ===========================================================================

Large language model (LLM) committees, ensembles of model instances whose
outputs are aggregated into a single decision, are increasingly deployed for
high-stakes tasks. Yet virtually all existing ensemble methods assume that
every participating agent is honest. We present \textsc{Equitas}, an
open-source benchmark that stress-tests aggregation strategies under
adversarial corruption in hierarchical multi-LLM committees. A three-level
hierarchy of LLM agents (class members, class leaders, and inter-class
judges) governs a simulated multi-stakeholder city while a configurable
fraction $\varepsilon$ of agents is corrupted by one of four adversary types
(selfish, coordinated, scheduled, deceptive). We evaluate nine aggregation
baselines plus an oracle across five experiment suites. The primary
full-hierarchy corruption sweep alone requires 103{,}776 OpenAI Batch API
requests. Our principal observations are: (i) at
$\varepsilon = 0.75$, supervisor rerank retains 99.6\% of its clean-setting
utility while non-adaptive methods lose over five percentage points; (ii) the
hierarchical architecture provides a 1.3--1.4\,pp utility advantage over flat
voting when $\varepsilon \geq 0.5$; and (iii) no single aggregation method
dominates across all corruption regimes among non-oracle baselines. All results are obtained with
GPT-4o-mini; whether these patterns generalise to other model families
remains an open question. Code and data are publicly available.\footnote{%
  Code: \url{https://github.com/akshan-main/Equitas}. \;
  Data: \url{https://huggingface.co/datasets/akshan-main/Equitas}.}

\end{abstract}

% ===========================================================================
\section{Introduction}
\label{sec:intro}
% ===========================================================================

Multi-agent LLM systems aggregate the outputs of several model instances to
improve reliability, coverage, and factual accuracy
\cite{wang2023selfconsistency, du2023debate, li2024more}. Applications
range from collaborative code generation to multi-perspective policy
deliberation. A growing body of work demonstrates that ensembling multiple
LLM responses can outperform any single response, mirroring classical results
on committee machines and mixture-of-experts.

A critical yet under-explored question is: \emph{what happens when some
agents in the committee are corrupted?} Corruption can arise through prompt
injection attacks, fine-tuning poisoning, supply-chain compromise of model
weights, or gradual model degradation after deployment. In any of these
scenarios, a subset of agents may produce outputs that are systematically
misleading, and the aggregation mechanism must cope with this adversarial
input.

Two lines of prior work are relevant but insufficient. On the theoretical
side, the multiplicative-weights (MW) update framework
\cite{arora2012multiplicative} provides regret bounds of the form
\begin{equation}
  R_T \;=\; O\!\left(\sqrt{T \ln n}\right),
  \label{eq:mw_regret}
\end{equation}
where $T$ is the number of rounds and $n$ is the number of experts. However,
MW assumes scalar losses over a fixed action set, assumptions that break
down when experts produce free-form natural-language recommendations. On the
empirical side, LLM ensemble methods such as
LLM-Blender~\cite{jiang2023llmblender} and
FrugalGPT~\cite{chen2023frugalgpt} demonstrate effective aggregation
strategies but assume all participating agents are honest.

We introduce \textsc{Equitas}, a benchmark designed to bridge this gap. Our
contributions are as follows:

\begin{enumerate}[leftmargin=*,itemsep=3pt]
\item A \textbf{hierarchical multi-LLM governance simulation}
  (Section~\ref{sec:environment}) in which class members, class leaders, and
  inter-class judges deliberate over a Plato-inspired multi-stakeholder city.
  The environment features inter-class tension by design: no single action
  simultaneously maximises utility for all stakeholder groups.

\item A \textbf{configurable adversary framework}
  (Section~\ref{sec:adversaries}) with four corruption strategies (selfish,
  coordinated, scheduled, and deceptive) applied at corruption rates
  $\varepsilon \in \{0, 0.25, 0.5, 0.75\}$.

\item A \textbf{systematic empirical evaluation} of ten aggregation methods
  across five experiment suites (Section~\ref{sec:results}), encompassing
  corruption sweeps, recovery dynamics, committee-size scaling, architectural
  comparisons, and welfare--fairness Pareto analysis.
\end{enumerate}

We emphasise several important caveats upfront. All experiments use a single
model (GPT-4o-mini). We do not claim that the observed rankings generalise to
other model families or to real-world deployments. The governance environment,
while designed to create meaningful inter-class trade-offs, is synthetic.
With 3--5 runs per condition, the 90\% bootstrap confidence intervals are
wide, and most pairwise differences between non-oracle methods are not
statistically significant.

% ===========================================================================
\section{Related Work}
\label{sec:related}
% ===========================================================================

\paragraph{Multi-agent LLM systems.}
Self-consistency \cite{wang2023selfconsistency} samples a single LLM $K$
times and takes the majority answer, improving chain-of-thought reasoning
without any inter-agent communication. LLM-Blender
\cite{jiang2023llmblender} goes further by training a pairwise ranker over
outputs from heterogeneous LLMs, followed by a generative fusion step.
Multi-agent debate \cite{du2023debate, liang2023encouraging} has agents
iteratively refine their answers through argumentation, improving factuality
on reasoning benchmarks. Li et al.\ \cite{li2024more} show that simply
scaling the number of agents can improve performance, a finding our scaling
experiments revisit under adversarial conditions. Critically, all of these
works assume that every participating agent is honest.

\paragraph{Robust aggregation theory.}
The multiplicative-weights update method \cite{arora2012multiplicative} is a
foundational online learning algorithm that tracks expert performance and
down-weights poor performers. Its regret guarantee
(Equation~\ref{eq:mw_regret}) holds even when a constant fraction of experts
are adversarial, provided losses are bounded and observable. Trimmed-mean and
median-of-means estimators \cite{lugosi2019mean} offer breakdown-point
guarantees for robust estimation under heavy-tailed distributions. Whether
these theoretical properties translate to LLM committees, where
``experts'' produce structured natural language and losses must be derived
post-hoc, is precisely the question Equitas is designed to investigate.

\paragraph{LLM robustness benchmarks.}
AdvBench \cite{zou2023universal} and HarmBench \cite{mazeika2024harmbench}
evaluate the robustness of individual models to adversarial prompts designed
to elicit harmful outputs. These benchmarks test \emph{input} robustness of
a \emph{single} model. In contrast, Equitas tests \emph{aggregation}
robustness when \emph{agents themselves} are adversarial, a fundamentally
different threat model that operates at the system level rather than the
model level.

\paragraph{Byzantine-tolerant distributed learning.}
Byzantine-tolerant gradient aggregation \cite{blanchard2017machine} addresses
adversarial workers in distributed stochastic gradient descent by filtering
or robustly aggregating gradient updates. The setting is analogous to ours
in that a fraction of workers are adversarial, but the medium differs:
Byzantine ML operates on gradient vectors in $\mathbb{R}^d$, whereas our
agents produce natural-language recommendations that must be parsed,
evaluated, and compared.

% ===========================================================================
\section{The Equitas Benchmark}
\label{sec:benchmark}
% ===========================================================================

\subsection{Governance Environment}
\label{sec:environment}

The benchmark is built around a simulated city inspired by Plato's
\emph{Republic}, comprising three citizen classes:
\begin{equation}
  \mathcal{C} = \{\text{guardian},\; \text{auxiliary},\; \text{producer}\}.
  \label{eq:classes}
\end{equation}
Each round $t \in \{1, \ldots, T\}$, a crisis is sampled from four axes:
resource scarcity, external threat, inequality, and economic instability. The
committee must select one of three candidate actions, $a \in \{1, 2, 3\}$.

For class $c$ and action $a$ at round $t$, we first compute a class-specific
score
\begin{equation}
  s_c(a,t) =
  \sum_{k=1}^{4} w^{\text{axis}}_{c,k}\, z_{t,k}
  \;+\;
  \sum_{d=1}^{4} w^{\text{policy}}_{c,d}\, p_{a,d},
  \label{eq:utility_score}
\end{equation}
where $z_{t,k}$ are crisis-axis values at round $t$ and $p_{a,d}$ are the
policy-dimension values for action $a$. Per-class utility is then
\begin{equation}
  u_c(a,t) = \sigma\!\left(s_c(a,t)\right),
  \label{eq:utility}
\end{equation}
where $\sigma$ is the sigmoid function. The city-wide utility is the arithmetic mean
over classes:
\begin{equation}
  U(a,t) = \frac{1}{|\mathcal{C}|} \sum_{c \in \mathcal{C}} u_c(a,t).
  \label{eq:city_utility}
\end{equation}

The class-specific axis and policy weights are designed so that actions exhibit
inter-class tension: the action that is optimal for one class is generally
suboptimal for another. This tension is what makes aggregation non-trivial
and fairness-relevant.

\paragraph{Fairness metrics.}
We track three fairness measures computed over the class-level utilities
$\{u_c(a,t)\}_{c \in \mathcal{C}}$:
\begin{itemize}[leftmargin=*,itemsep=2pt]
  \item \textbf{Jain's fairness index:}
    $J = \left(\sum_c u_c\right)^2 \Big/ \left(|\mathcal{C}| \sum_c u_c^2\right)$,
    ranging from $1/|\mathcal{C}|$ (maximally unfair) to $1$ (perfectly fair).
  \item \textbf{Worst-group utility:} $\min_{c \in \mathcal{C}} u_c(a,t)$.
  \item \textbf{Unfairness gap:}
    $\max_{c} u_c(a,t) - \min_{c} u_c(a,t)$.
\end{itemize}

\subsection{Hierarchical Architecture}
\label{sec:hierarchy}

The committee is organised into a two-level hierarchy
(Figure~\ref{fig:corruption_sweep} header):

\paragraph{Level 1 (intra-class aggregation).}
Each class $c \in \mathcal{C}$ has $N = 7$ members (by default) who
independently recommend actions via LLM calls. A class-level aggregator
selects the preferred action from these recommendations, and the class
\emph{leader} produces a proposal consisting of the chosen action and a
natural-language rationale.

\paragraph{Level 2 (inter-class aggregation).}
The three class proposals are forwarded to a panel of $M = 5$ judges, each
of which independently evaluates them via an LLM call and votes for one. A
\emph{governor}, a purely algorithmic aggregator with no LLM call, makes
the final decision based on the judges' votes.

This architecture requires approximately $3N + 3 + M = 29$ LLM calls per
round (21 members, 3 leaders, 5 judges), with the governor being
deterministic and auditable.

\paragraph{Two evaluation protocols.}
We distinguish two protocols for evaluating aggregation methods:
\begin{itemize}[leftmargin=*,itemsep=2pt]
  \item \textbf{Full-Hierarchy (FH):} The aggregation method under test
    controls \emph{both} the intra-class (Level~1) and inter-class (Level~2)
    aggregation. This is the primary evaluation protocol.
  \item \textbf{Governor-Only (GO):} A fixed trimmed-vote rule is used at
    Level~1, and only the Level~2 aggregation varies. This serves as a
    diagnostic ablation (see Appendix~\ref{sec:go_ablation}).
\end{itemize}

\subsection{Adversary Types}
\label{sec:adversaries}

A fraction $\varepsilon \in \{0, 0.25, 0.5, 0.75\}$ of agents are replaced
by adversaries. We implement four adversary types of increasing
sophistication:

\begin{enumerate}[leftmargin=*,itemsep=3pt]
\item \textbf{Selfish.} The corrupted agent recommends the action that
  maximises its own class's utility $u_c(a)$, disregarding the city-wide
  objective. Because selfish agents still optimise \emph{some} legitimate
  utility, this is the mildest form of corruption.

\item \textbf{Coordinated.} All corrupted agents collude to push the same
  action, specifically the action that is worst for city-wide utility.
  This simulates a coordinated attack where adversaries agree on a common
  strategy.

\item \textbf{Adaptive (scheduled).} Corrupted agents behave honestly for
  the first $K$ rounds to build trust (accumulating high MW weights), then
  switch to adversarial behaviour. This directly targets the trust
  mechanisms of adaptive aggregators.

\item \textbf{Deceptive.} Corrupted agents recommend the wrong action
  \emph{and} generate a persuasive LLM-written rationale to justify it.
  This is the most sophisticated attack, targeting the judge layer's ability
  to evaluate proposal quality.
\end{enumerate}

\subsection{Aggregation Methods}
\label{sec:aggregators}

We evaluate nine methods plus an oracle ceiling, summarised in
Table~\ref{tab:methods}. These span three categories: non-adaptive methods
that treat all agents equally, adaptive methods that learn from past
performance, and oracle baselines that use privileged information.

The central adaptive method is multiplicative weights (MW), where each agent
$i$ maintains a weight $w_i^{(t)}$ updated after each round:
\begin{equation}
  w_i^{(t+1)} = w_i^{(t)} \cdot \exp\!\left(-\eta \, \ell_i^{(t)}\right),
  \label{eq:mw_update}
\end{equation}
where $\eta$ is the learning rate and $\ell_i^{(t)}$ is the per-agent loss.
The loss is a weighted combination of welfare and fairness components:
\begin{equation}
  \ell_i^{(t)} = \alpha \, \ell_{\text{welfare}}^{(t)}
                + \beta \, \ell_{\text{fairness}}^{(t)},
  \label{eq:mw_loss}
\end{equation}
with defaults $\alpha = 1$, $\beta = 0.5$, $\eta = 1$. In the FH protocol,
MW is applied at both the intra-class (member weights) and inter-class
(judge weights) levels.

We define the \textbf{robustness ratio} of an aggregation method as:
\begin{equation}
  \rho = \frac{U(\varepsilon = 0.75)}{U(\varepsilon = 0)},
  \label{eq:robustness}
\end{equation}
i.e., the fraction of clean-setting utility retained at the highest
corruption rate.

\begin{table}[!ht]
\centering
\scriptsize
\caption{Aggregation methods evaluated in Equitas. Non-adaptive methods use
  fixed, equal weights. Adaptive methods adjust weights based on past
  performance. Oracle methods use privileged information.}
\label{tab:methods}
\begin{tabular}{@{}llp{3.5cm}@{}}
\toprule
Method & Category & Description \\
\midrule
Majority vote     & Non-adaptive & Equal-weight plurality \\
Self-consistency  & Non-adaptive & Same LLM sampled $K{=}5$ times \\
Random dictator   & Non-adaptive & Pick one agent uniformly \\
EMA trust         & Adaptive   & Exponential moving avg.\ of $1 - \ell_i$ \\
Trimmed vote      & Adaptive   & Drop worst 20\%, then vote \\
Conf.\ weighted   & Adaptive   & Harmonic weights from cumulative loss \\
MW                & Adaptive   & $w_i \gets w_i \exp(-\eta \ell_i)$ \\
Supervisor rerank & Adaptive   & Follow-the-leader (delegate to lowest-loss agent) \\
Oracle UB         & Oracle     & Best-of-$K$ with oracle weights \\
Oracle            & Oracle     & Hindsight-optimal action each round \\
\bottomrule
\end{tabular}
\end{table}

% ===========================================================================
\section{Experimental Setup}
\label{sec:experiments}
% ===========================================================================

All experiments use GPT-4o-mini (temperature 0.7, max 512 tokens). We use
the OpenAI Batch API for large sweep runs and asynchronous online calls for
auxiliary suites (recovery, scaling, architecture, Pareto), with full
record/replay logs in both cases. Each condition is repeated across 3--5 independent runs
with different random seeds. We report trimmed means and 90\% bootstrap
confidence intervals throughout.

\subsection{Experiment 1: Corruption Sweep}
\label{sec:exp_sweep}

The primary experiment crosses corruption rate $\varepsilon \in \{0, 0.25,
0.5, 0.75\}$ with all four adversary types, yielding $4 \times 4 = 16$
conditions. Each aggregator is evaluated across 5 runs of $T = 40$ rounds
each, for a total of 80 runs per aggregator.

\subsection{Experiment 2: Recovery from Mid-Run Corruption}
\label{sec:exp_recovery}

To study how aggregation methods respond to sudden adversarial onset, we run
$T = 40$ rounds with $\varepsilon = 0$ for the first 20 rounds, then switch
to $\varepsilon = 0.5$ (scheduled adversary) for rounds 21--40. This tests
whether adaptive methods can detect and recover from corruption after having
built trust during a clean phase.

\subsection{Experiment 3: Committee-Size Scaling}
\label{sec:exp_scaling}

We vary the number of members per class, $N \in \{3, 5, 7, 10\}$, at fixed
corruption $\varepsilon = 0.5$ with selfish adversaries. This tests
whether larger committees dilute the influence of adversarial agents or
whether prompt crowding degrades leader quality.

\subsection{Experiment 4: Hierarchical vs.\ Flat Architecture}
\label{sec:exp_hvf}

We compare the full hierarchy (members $\to$ leaders $\to$ judges $\to$
governor) against a flat architecture in which all $3N = 21$ agents vote
directly, with no intermediate leader or judge layer. Both architectures use
all ten aggregators at $\varepsilon \in \{0, 0.25, 0.5, 0.75\}$.

\subsection{Experiment 5: Welfare--Fairness Pareto Sweep}
\label{sec:exp_pareto}

We sweep the MW loss parameters
$(\alpha, \beta) \in \{0, 0.25, 0.5, 0.75, 1.0\}^2$ (25 grid points) to
trace the Pareto frontier of the welfare--fairness trade-off. This reveals
which combinations of welfare weight and fairness weight produce
Pareto-optimal outcomes, and how sensitive MW is to these hyperparameters.

% ===========================================================================
\section{Results}
\label{sec:results}
% ===========================================================================

\subsection{Overall Leaderboard}
\label{sec:leaderboard}

Table~\ref{tab:leaderboard} presents the aggregate leaderboard under the FH
protocol, averaging across all corruption rates and adversary types. The
oracle, which selects the hindsight-optimal action each round, achieves a
mean utility of 0.466 and has a confidence interval that is nearly disjoint
from the rest (with slight overlap with supervisor rerank). Supervisor rerank (0.449) and MW (0.446)
occupy the next two positions, but their confidence intervals overlap both
with each other and with several lower-ranked methods. Ranks 4 through 10
span a range of only 3\,pp (0.436--0.439), which is well within the
confidence intervals, effectively a statistical tie.

Fairness scores (Jain's index) are remarkably uniform across methods,
ranging from 0.923 to 0.935. This indicates that, in this environment,
aggregation method choice has a much larger impact on welfare than on
fairness.

\begin{table}[!ht]
\centering
\scriptsize
\caption{FH leaderboard averaged over all conditions. Confidence intervals
  for ranks 2--10 overlap; most pairwise differences between non-oracle
  methods are \emph{not} statistically significant.}
\label{tab:leaderboard}
\begin{tabular}{@{}clcccc@{}}
\toprule
Rank & Aggregator & Utility & 90\% CI & Fairness & Robustness \\
\midrule
1  & Oracle       & .4655 & [.456,\,.475] & .923 & .995 \\
2  & Supervisor   & .4488 & [.439,\,.459] & .935 & .996 \\
3  & MW           & .4458 & [.436,\,.456] & .935 & .955 \\
4  & Random       & .4387 & [.429,\,.449] & .933 & .930 \\
5  & Conf.\ wt    & .4384 & [.429,\,.448] & .934 & .890 \\
6  & Oracle UB    & .4378 & [.428,\,.448] & .934 & .890 \\
7  & EMA          & .4371 & [.427,\,.447] & .934 & .884 \\
8  & Trimmed      & .4371 & [.427,\,.447] & .934 & .885 \\
9  & Majority     & .4364 & [.427,\,.446] & .933 & .884 \\
10 & Self-cons    & .4357 & [.426,\,.445] & .933 & .887 \\
\bottomrule
\end{tabular}
\end{table}

% ---------------------------------------------------------------------------
\subsection{Corruption Degradation}
\label{sec:corruption_sweep}

\begin{figure*}[!ht]
\centering
\includegraphics[width=\textwidth]{figures/fig_corruption_sweep.png}
\caption{City utility as a function of corruption rate $\varepsilon$,
  broken out by adversary type (FH protocol). Each curve corresponds to one
  aggregation method; shaded bands indicate 90\% bootstrap CIs. At
  $\varepsilon = 0.75$, supervisor rerank (cyan) maintains near-constant
  utility while non-adaptive methods (majority, self-consistency) drop
  sharply. The separation is most pronounced under coordinated and deceptive
  adversaries.}
\label{fig:corruption_sweep}
\end{figure*}

Table~\ref{tab:corruption} and Figure~\ref{fig:corruption_sweep} present
the core corruption-sweep results. Three regimes emerge.

At \textbf{low corruption} ($\varepsilon \leq 0.25$), all methods perform
within approximately 1\,pp of each other (utility range 0.446--0.453).
Adversarial agents are too few to overcome honest majorities, and even
non-adaptive methods function adequately.

At \textbf{moderate corruption} ($\varepsilon = 0.50$), a gap begins to
open. Non-adaptive methods start to degrade (e.g., majority vote drops from
0.452 to 0.444), while MW and supervisor maintain performance.

At \textbf{high corruption} ($\varepsilon = 0.75$), the methods sharply
diverge. Non-adaptive methods collapse to approximately 0.400 (a drop of
5.2\,pp from their clean-setting utility), while supervisor rerank retains
0.448, yielding a robustness ratio $\rho = 0.996$. MW degrades more
moderately to 0.432 ($\rho = 0.955$). Random dictator achieves $\rho =
0.930$, explained by its uniform selection: at $\varepsilon = 0.75$, it
still selects an honest agent 25\% of the time, providing a baseline
corruption tolerance.

\begin{table}[!ht]
\centering
\scriptsize
\caption{Utility by corruption rate (FH, averaged over adversary types).
  Robustness $\rho$ is the ratio of $\varepsilon = 0.75$ to
  $\varepsilon = 0$ utility (Equation~\ref{eq:robustness}).}
\label{tab:corruption}
\begin{tabular}{@{}lccccc@{}}
\toprule
Aggregator & $\varepsilon{=}0$ & $0.25$ & $0.50$ & $0.75$ & $\rho$ \\
\midrule
Oracle     & .4662 & .4661 & .4660 & .4637 & .995 \\
Supervisor & .4495 & .4482 & .4502 & .4475 & .996 \\
MW         & .4519 & .4497 & .4501 & .4316 & .955 \\
Conf.\ wt  & .4519 & .4496 & .4496 & .4023 & .890 \\
EMA        & .4520 & .4498 & .4470 & .3997 & .884 \\
Trimmed    & .4516 & .4500 & .4471 & .3996 & .885 \\
Majority   & .4522 & .4494 & .4442 & .3997 & .884 \\
Self-cons  & .4517 & .4493 & .4412 & .4007 & .887 \\
Oracle UB  & .4528 & .4502 & .4454 & .4029 & .890 \\
Random     & .4508 & .4464 & .4381 & .4193 & .930 \\
\bottomrule
\end{tabular}
\end{table}

% ---------------------------------------------------------------------------
\subsection{Adversary Type Analysis}
\label{sec:adversary_analysis}

Table~\ref{tab:adversary} breaks down utility by adversary type, averaging
over all corruption rates. Several patterns are worth noting.

\textbf{Selfish adversaries} are the least damaging. Because selfish agents
still optimise a legitimate (class-level) utility, their recommendations are
not entirely destructive, and all methods perform within a narrow band
(0.447--0.450 for non-oracle methods).

\textbf{Coordinated and deceptive adversaries} are the most damaging,
causing 2--3\,pp drops for non-adaptive methods relative to the selfish
condition. The coordinated attack is effective because all corrupted agents
push the same harmful action, creating a strong adversarial signal.
Deceptive adversaries compound this with persuasive rationales that can
mislead the judge layer.

An interesting \textbf{crossover} occurs between supervisor and MW.
Supervisor rerank outperforms MW against coordinated attacks by 0.6\,pp
(0.452 vs.\ 0.446), likely because the supervisor LLM can detect
semantically implausible coordinated proposals. Conversely, MW outperforms
supervisor against selfish adversaries by 0.3\,pp (0.449 vs.\ 0.447),
where the adversarial proposals are more plausible and the weight-based
filtering of MW is more effective.

\begin{table}[!ht]
\centering
\scriptsize
\caption{Utility by adversary type (FH, all $\varepsilon$ averaged).}
\label{tab:adversary}
\begin{tabular}{@{}lcccc@{}}
\toprule
Aggregator & Selfish & Coord. & Sched. & Decep. \\
\midrule
Oracle     & .462 & .467 & .468 & .465 \\
Supervisor & .447 & .452 & .450 & .447 \\
MW         & .449 & .446 & .446 & .442 \\
Conf.\ wt  & .450 & .433 & .439 & .432 \\
EMA        & .450 & .432 & .438 & .429 \\
Trimmed    & .449 & .433 & .438 & .429 \\
Majority   & .450 & .431 & .437 & .428 \\
Self-cons  & .450 & .430 & .435 & .428 \\
Oracle UB  & .450 & .432 & .439 & .430 \\
Random     & .447 & .434 & .440 & .434 \\
\bottomrule
\end{tabular}
\end{table}

% ---------------------------------------------------------------------------
\subsection{Regime-Specific Winners}
\label{sec:regime}

A recurring theme in our results is that \emph{no single non-oracle method
dominates across all regimes}. Table~\ref{tab:regime} shows
selected regime cells with the strongest non-oracle contender and its
utility gap to oracle.

At $\varepsilon = 0$ (no corruption), non-oracle winners are typically
non-adaptive or lightly adaptive methods (e.g., self-consistency, EMA,
oracle upper bound), and margins between non-oracle methods are tiny
(often $< 0.001$). At $\varepsilon \geq 0.5$, supervisor rerank wins in
5 out of 8 high-corruption cells for the welfare criterion, with the largest
separation appearing at $\varepsilon = 0.75$ under coordinated, deceptive,
and scheduled adversaries.

\begin{table}[!ht]
\centering
\scriptsize
\caption{Selected welfare regimes: strongest non-oracle method and its gap to
  oracle.}
\label{tab:regime}
\begin{tabular}{@{}llll@{}}
\toprule
Adversary & $\varepsilon$ & Best non-oracle & Oracle gap \\
\midrule
Coordinated & 0.00 & Self-Cons  & .014 \\
Coordinated & 0.75 & Supervisor & .014 \\
Deceptive   & 0.50 & MW         & .019 \\
Deceptive   & 0.75 & Supervisor & .017 \\
Scheduled   & 0.75 & Supervisor & .020 \\
Selfish     & 0.75 & Conf-Wt    & .010 \\
\bottomrule
\end{tabular}
\end{table}

% ---------------------------------------------------------------------------
\subsection{Recovery from Corruption Onset}
\label{sec:recovery}

Figure~\ref{fig:recovery} and Table~\ref{tab:recovery} present the recovery
experiment, in which corruption is introduced at round~20 of a 40-round
simulation.

\begin{figure}[!ht]
\centering
\includegraphics[width=\columnwidth]{figures/fig_recovery.png}
\caption{\textbf{Left:} Per-round city utility over 40 rounds. The red
  dashed line marks corruption onset at round~20. All methods degrade;
  per-round noise is typical for small committees. \textbf{Right:} MW
  inter-class judge weights over time. After corruption onset, MW
  progressively down-weights corrupted judges, though convergence is slow.}
\label{fig:recovery}
\end{figure}

\begin{table}[!ht]
\centering
\scriptsize
\caption{Recovery from corruption onset at round~20 ($\varepsilon = 0.5$,
  scheduled), for selected aggregators. ``Late U'' is the mean utility over
  the final 5 rounds. Recovery gap $= $ Late U $-$ Pre-onset U.}
\label{tab:recovery}
\begin{tabular}{@{}lcccc@{}}
\toprule
Aggregator & Pre U & Post U & Late U & Rec.\ Gap \\
\midrule
Oracle     & .462 & .447 & .443 & $-$.019 \\
MW         & .451 & .427 & .419 & $-$.032 \\
Supervisor & .447 & .425 & .414 & $-$.032 \\
EMA        & .451 & .427 & .424 & $-$.027 \\
Trimmed    & .451 & .430 & .425 & $-$.026 \\
Majority   & .451 & .422 & .422 & $-$.029 \\
Self-cons  & .452 & .404 & .401 & $-$.051 \\
Random     & .451 & .413 & .398 & $-$.052 \\
\bottomrule
\end{tabular}
\end{table}

A notable finding is that the simplest robust methods, trimmed vote
and EMA, show the smallest recovery gaps ($-$0.026 and $-$0.027,
respectively), outperforming both MW and supervisor ($-$0.032 each). This is
somewhat counterintuitive. The explanation lies in the \emph{trust
reservoir} problem: during the honest first 20 rounds, MW assigns high
weights to all agents, including those that will later become adversarial.
After corruption onset, MW must gradually erode these accumulated weights
through the exponential update (Equation~\ref{eq:mw_update}), a process
that is inherently slow when the initial trust is high. Trimmed vote avoids
this issue entirely by simply dropping the worst performers without
maintaining a trust history.

Self-consistency and random dictator show the largest recovery gaps ($-$0.051
and $-$0.052), as neither has any mechanism to down-weight corrupted agents.

% ---------------------------------------------------------------------------
\subsection{Committee-Size Scaling}
\label{sec:scaling}

Figure~\ref{fig:scaling} and Table~\ref{tab:scaling} examine how committee
size affects performance under corruption.

\begin{figure}[!ht]
\centering
\includegraphics[width=\columnwidth]{figures/fig_scaling.png}
\caption{\textbf{Left:} City utility vs.\ number of members per class $N$.
  \textbf{Right:} Jain's fairness index vs.\ $N$. Both metrics exhibit an
  inverted-U, peaking at $N = 7$ and degrading at $N = 10$.}
\label{fig:scaling}
\end{figure}

\begin{table}[!ht]
\centering
\scriptsize
\caption{Utility by committee size ($\varepsilon = 0.5$, selfish), shown for
  selected aggregators.}
\label{tab:scaling}
\begin{tabular}{@{}lcccc@{}}
\toprule
Aggregator & $N{=}3$ & $N{=}5$ & $N{=}7$ & $N{=}10$ \\
\midrule
Oracle     & .463 & .476 & .478 & .469 \\
MW         & .446 & .464 & .466 & .455 \\
Supervisor & .447 & .459 & .465 & .450 \\
Majority   & .447 & .463 & .466 & .458 \\
\bottomrule
\end{tabular}
\end{table}

All methods exhibit a consistent inverted-U pattern, with performance peaking
at $N = 7$ and declining at $N = 10$. The improvement from $N = 3$ to
$N = 7$ is expected: larger committees dilute the adversarial fraction,
giving the aggregator more honest signal to work with. The degradation at
$N = 10$ is more surprising and likely reflects \emph{prompt crowding}: the
class leader must synthesise recommendations from 10 members into a single
proposal, and the longer prompt may degrade LLM generation quality.

Importantly, \emph{all} methods, including the oracle, show this
inverted-U, suggesting it is a property of the environment and the LLM's
prompt-processing capacity rather than of any specific aggregation algorithm.

% ---------------------------------------------------------------------------
\subsection{Hierarchical vs.\ Flat Architecture}
\label{sec:hvf}

Tables~\ref{tab:hvf} and \ref{tab:hvf_detail} compare the hierarchical and
flat architectures.

\begin{table}[!ht]
\centering
\scriptsize
\caption{Hierarchical vs.\ flat (averaged over all 10 aggregators).
  $\Delta$U is the hierarchy advantage.}
\label{tab:hvf}
\begin{tabular}{@{}ccccc@{}}
\toprule
$\varepsilon$ & Hier.\ U & Flat U & $\Delta$U & $\Delta$Fair \\
\midrule
0.00 & .4631 & .4631 & $-$.000 & $+$.001 \\
0.25 & .4594 & .4598 & $-$.000 & $-$.001 \\
0.50 & .4576 & .4435 & $+$.014 & $+$.007 \\
0.75 & .4723 & .4592 & $+$.013 & $+$.006 \\
\bottomrule
\end{tabular}
\end{table}

\begin{table}[!ht]
\centering
\scriptsize
\caption{Per-aggregator hierarchy advantage at $\varepsilon = 0.75$. Every
  method benefits from the hierarchical structure.}
\label{tab:hvf_detail}
\begin{tabular}{@{}lccc@{}}
\toprule
Aggregator & Hier.\ U & Flat U & Gap \\
\midrule
Oracle     & .478 & .466 & $+$.013 \\
MW         & .474 & .459 & $+$.015 \\
Conf-Wt    & .474 & .459 & $+$.015 \\
EMA        & .475 & .459 & $+$.016 \\
Supervisor & .467 & .454 & $+$.013 \\
Majority   & .474 & .460 & $+$.014 \\
Self-cons  & .474 & .462 & $+$.012 \\
Oracle UB  & .472 & .457 & $+$.015 \\
Trimmed    & .470 & .460 & $+$.010 \\
Random     & .463 & .454 & $+$.008 \\
\bottomrule
\end{tabular}
\end{table}

At low corruption ($\varepsilon \leq 0.25$), the two architectures are
essentially equivalent; the hierarchy neither helps nor hurts when most
agents are honest. At $\varepsilon \geq 0.5$, however, the hierarchy
provides a consistent advantage of 1.3--1.4\,pp in utility and 0.6--0.7\,pp
in fairness. Table~\ref{tab:hvf_detail} shows that this advantage is
present for \emph{every} aggregation method, from oracle ($+$1.3\,pp) to
random dictator ($+$0.8\,pp). This universality suggests that the benefit is
a \emph{structural} property of the class-leader bottleneck: by forcing
recommendations through a leader, the hierarchy acts as an information
filter that attenuates adversarial noise, rather than a property of any
particular aggregation algorithm.

% ---------------------------------------------------------------------------
\subsection{Welfare--Fairness Pareto Analysis}
\label{sec:pareto}

Figure~\ref{fig:pareto} and Table~\ref{tab:pareto} present the Pareto sweep
over MW loss parameters $(\alpha, \beta)$.

\begin{figure}[!ht]
\centering
\includegraphics[width=\columnwidth]{figures/fig_pareto.png}
\caption{Welfare--fairness scatter for the MW Pareto sweep. Each point
  represents one $(\alpha, \beta)$ configuration averaged over seeds. The
  Pareto-optimal grid point is $(\alpha = 1.0, \beta = 0.25)$, achieving the
  highest utility (0.479) and fairness (0.939) simultaneously.}
\label{fig:pareto}
\end{figure}

\begin{table}[!ht]
\centering
\scriptsize
\caption{Selected points from the MW Pareto sweep. The Pareto-optimal point
  dominates all other grid points on both welfare and fairness axes.}
\label{tab:pareto}
\begin{tabular}{@{}ccccc@{}}
\toprule
$\alpha$ & $\beta$ & Utility & Fairness & Worst-Grp \\
\midrule
\multicolumn{5}{l}{\emph{Pareto-optimal:}} \\
1.00 & 0.25 & \textbf{.479} & \textbf{.939} & \textbf{.334} \\
\midrule
1.00 & 1.00 & .462 & .935 & .316 \\
1.00 & 0.00 & .423 & .923 & .270 \\
0.00 & 0.00 & .432 & .928 & .284 \\
\midrule
\multicolumn{5}{l}{\emph{Default ($\alpha = 1$, $\beta = 0.5$):}} \\
1.00 & 0.50 & .451 & .927 & .296 \\
\bottomrule
\end{tabular}
\end{table}

The Pareto-optimal grid point is $(\alpha = 1.0, \beta = 0.25)$, achieving
0.479 utility and 0.939 fairness, dominating all other grid points on both
axes. This reveals a non-obvious interaction: a moderate fairness penalty
($\beta = 0.25$) is not only good for fairness but also \emph{improves
welfare} relative to the welfare-only configuration ($\beta = 0$, which
yields only 0.423). The mechanism is that a small fairness pressure
discourages actions that are excellent for one class but harmful to others,
steering the committee towards compromises that happen to achieve higher
city-wide utility.

The default configuration ($\alpha = 1$, $\beta = 0.5$) achieves 0.451
utility, which is 2.8\,pp below the Pareto-optimal point. This suggests that
the commonly used default is suboptimal and that even simple hyperparameter
tuning of the loss weights can yield meaningful improvements.

% ===========================================================================
\section{Analysis and Discussion}
\label{sec:discussion}
% ===========================================================================

\paragraph{Why does supervisor rerank work so well?}
Supervisor rerank achieves the highest robustness ratio (0.996) by a
substantial margin. In our implementation, supervisor rerank is an
algorithmic follow-the-leader rule (no extra LLM): each round it delegates to
the currently lowest cumulative-loss agent. This creates strong performance
when quality separation is stable and corruption is persistent, because the
aggregator rapidly commits to a high-performing agent. A key caveat is
structural: all governor-level aggregators are algorithmic and non-corruptible
in our threat model.

\paragraph{The MW--supervisor trade-off.}
At $\varepsilon = 0$, MW slightly outperforms supervisor (0.452
vs.\ 0.450), as aggressive follow-the-leader commitment offers little benefit
when most agents are already honest. The crossover occurs at
$\varepsilon = 0.75$, where supervisor leads by 1.6\,pp (0.448 vs.\ 0.432).
This suggests a practical deployment strategy: use MW when the corruption
threat is low, and prefer supervisor rerank when the threat is elevated.

\paragraph{The GO vs.\ FH ablation.}
In the GO protocol, all non-oracle methods cluster at approximately 0.401
at $\varepsilon = 0.75$, including MW and supervisor (Appendix~\ref{sec:go_ablation}). This occurs because the fixed Level-1 aggregator
prevents these methods from exercising their adaptive advantages at the
intra-class level. Under FH, supervisor gains $+$4.6\,pp and MW gains
$+$3.1\,pp relative to GO. This confirms that the benefits of adaptive
aggregation are \emph{primarily realised at the intra-class level}, where
member-level corruption is directly observable through per-agent loss
tracking.

\paragraph{Trust reservoir and adaptive lag.}
The recovery experiment reveals that MW's theoretical strength, tracking
individual agent performance, becomes a liability when corruption has a
sudden onset. The weights accumulated during the honest phase act as a
``trust reservoir'' that corrupted agents exploit. After onset, MW
must erode these weights via repeated multiplicative down-weighting
(Equation~\ref{eq:mw_update}), a process that converges as
$w_i^{(t)} \propto e^{-\eta \sum_s \ell_i^{(s)}}$, which requires multiple
rounds of high loss to substantially reduce the weight. Stateless methods
like trimmed vote, which simply discard the worst performers without
considering history, are immune to this lag.

% ===========================================================================
\section{Limitations}
\label{sec:limitations}
% ===========================================================================

\paragraph{Single model.}
All results are obtained with GPT-4o-mini. The observed rankings may not hold
for larger models (which may be harder to mislead), smaller models (which may
be more susceptible), or heterogeneous committees mixing different model
families.

\paragraph{Synthetic environment.}
The governance environment uses sigmoid utility functions and a 3-action
space, which simplifies real-world multi-stakeholder decision-making. The
inter-class tensions are hand-designed; natural tasks may exhibit different
correlation structures.

\paragraph{Statistical power.}
With 3--5 runs per condition, the 90\% bootstrap confidence intervals are
wide. Most pairwise comparisons between non-oracle methods are not
statistically significant. We report all results to enable meta-analysis
but caution against over-interpreting small ranking differences.

\paragraph{Fixed adversary strategies.}
Our four adversary types use fixed strategies. Real-world adversaries may
employ mixed strategies, time-varying corruption rates, or strategically
target specific agents based on observed system behaviour.

\paragraph{Computational cost.}
The main FH corruption sweep requires 103{,}776 OpenAI Batch API requests,
and the full suite requires substantially more. We mitigate cost through
batching and a record-replay architecture, but a full re-run remains
non-trivial.

\paragraph{Algorithmic governor assumption.}
All governor-level aggregators are purely algorithmic and are not directly
corrupted in our current threat model. Robustness under direct corruption of
the aggregation mechanism itself remains open.

% ===========================================================================
\section{Conclusion}
\label{sec:conclusion}
% ===========================================================================

We have presented \textsc{Equitas}, a benchmark for evaluating the robustness
of aggregation methods in hierarchical multi-LLM committees under adversarial
corruption. Across five experiment suites, we
observe the following:

\begin{enumerate}[leftmargin=*,itemsep=3pt]
\item \textbf{Adaptive aggregation provides meaningful robustness.}
  Non-adaptive methods (majority vote, self-consistency) collapse to
  approximately 0.400 utility at $\varepsilon = 0.75$, a loss of 5.2\,pp
  from their clean-setting performance. MW retains 0.432 and supervisor
  retains 0.448.

\item \textbf{Supervisor rerank is the most robust method in our evaluation,}
  retaining 99.6\% of clean utility at $\varepsilon = 0.75$; it is a purely
  algorithmic rule in our implementation.

\item \textbf{Hierarchy provides a structural robustness benefit} of
  1.3--1.4\,pp at $\varepsilon \geq 0.5$, benefiting all aggregation methods
  equally through the class-leader information bottleneck.

\item \textbf{No single non-oracle method dominates all regimes.} Different
  aggregators win under different $(\varepsilon, \text{adversary})$
  combinations, and most pairwise differences are within confidence
  intervals.

\item \textbf{Stateless robust methods recover faster from sudden
  corruption} than stateful methods like MW, which suffer from a trust
  reservoir that takes time to drain.
\end{enumerate}

All code, experiment configurations, raw results, and LLM conversation logs
are publicly released to facilitate reproduction and extension. Key
directions for future work include evaluating across model families, testing
corruptible governor-level aggregation mechanisms, incorporating natural-language tasks (e.g.,
GSM8K), and developing adaptive adversaries that respond to the aggregation
strategy.

% ===========================================================================
\appendix
% ===========================================================================

\section{GO vs.\ FH Ablation Details}
\label{sec:go_ablation}

Table~\ref{tab:go_fh} compares the GO and FH protocols. In GO mode, a fixed
trimmed-vote rule at Level~1 masks the differences between aggregation
methods: all non-oracle methods achieve approximately 0.438 utility overall
and cluster at 0.401 at $\varepsilon = 0.75$. FH unlocks the adaptive
advantages of MW and supervisor by allowing them to operate at the
intra-class level, where per-agent corruption is directly observable.

\begin{table}[H]
\centering
\scriptsize
\caption{GO vs.\ FH protocol comparison. ``All'' is averaged over all
  conditions; ``$\varepsilon = 0.75$'' isolates high corruption.}
\label{tab:go_fh}
\begin{tabular}{@{}lcccc@{}}
\toprule
Aggregator & GO (all) & FH (all) & GO (.75) & FH (.75) \\
\midrule
Oracle     & .466 & .466 & .464 & .464 \\
MW         & .438 & .446 & .401 & .432 \\
Supervisor & .438 & .449 & .401 & .448 \\
EMA        & .438 & .437 & .401 & .400 \\
Trimmed    & .438 & .437 & .401 & .400 \\
Majority   & .438 & .436 & .401 & .400 \\
Self-cons  & .438 & .436 & .401 & .401 \\
\bottomrule
\end{tabular}
\end{table}

The key insight is that supervisor rerank gains $+$4.6\,pp and MW gains
$+$3.1\,pp when moving from GO to FH at $\varepsilon = 0.75$, while
non-adaptive methods show negligible change. This confirms that adaptive
aggregation at the intra-class level is the primary driver of robustness in
the hierarchical architecture.

Table~\ref{tab:go_fh_high} provides the full per-aggregator breakdown at
$\varepsilon = 0.75$, including all ten methods. The pattern is stark: under
GO, all non-oracle methods compress into a 0.4\,pp band (0.4006--0.4039),
whereas under FH, the spread widens to 4.8\,pp (0.3996--0.4475). Random
dictator is the only non-adaptive method that benefits from FH ($+$1.9\,pp),
because its uniform selection occasionally delegates to the honest minority,
an effect amplified when FH allows class-level filtering.

\begin{table}[H]
\centering
\scriptsize
\caption{Per-aggregator GO vs.\ FH comparison at $\varepsilon = 0.75$
  (all adversary types averaged). Gap = FH $-$ GO.}
\label{tab:go_fh_high}
\begin{tabular}{@{}lccc@{}}
\toprule
Aggregator & GO (.75) & FH (.75) & Gap \\
\midrule
Oracle     & .4637 & .4637 & $+$.0000 \\
Supervisor & .4013 & .4475 & $+$.0462 \\
MW         & .4009 & .4316 & $+$.0306 \\
Random     & .4006 & .4193 & $+$.0187 \\
Conf-Wt    & .4009 & .4023 & $+$.0013 \\
Self-Cons  & .4011 & .4007 & $-$.0005 \\
Oracle UB  & .4039 & .4029 & $-$.0010 \\
EMA        & .4008 & .3997 & $-$.0010 \\
Trimmed    & .4009 & .3996 & $-$.0013 \\
Majority   & .4011 & .3997 & $-$.0014 \\
\bottomrule
\end{tabular}
\end{table}

\section{Cross-Experiment Summary}
\label{sec:grand}

Table~\ref{tab:grand} summarises the GO vs.\ FH comparison across all five
experiment suites. FH outperforms GO in three of five suites. The reversal
in the Pareto suite occurs because extreme $(\alpha, \beta)$ configurations
can destabilise MW's intra-class weighting, an effect unique to FH where MW
operates at both levels.

\begin{table}[H]
\centering
\scriptsize
\caption{GO vs.\ FH protocol comparison across all experiment suites.}
\label{tab:grand}
\begin{tabular}{@{}lcccc@{}}
\toprule
Experiment & GO U & FH U & $\Delta$U & $\Delta$Fair \\
\midrule
Sweep       & .441 & .442 & $+$.002 & $+$.001 \\
Recovery    & .439 & .438 & $-$.001 & $+$.006 \\
Scaling     & .423 & .459 & $+$.036 & $+$.007 \\
Hier/Flat   & .421 & .463 & $+$.042 & $+$.006 \\
Pareto (MW) & .456 & .427 & $-$.030 & $-$.009 \\
\bottomrule
\end{tabular}
\end{table}

\section{Fairness Regime Analysis}
\label{sec:fairness_regimes}

The main text focuses primarily on welfare (city utility). Here we examine
which aggregators perform best under each regime when evaluated on fairness
criteria. Table~\ref{tab:fairness_regime} reports the strongest non-oracle
method for Jain's fairness index across selected regimes, while
Table~\ref{tab:wg_regime} does the same for worst-group utility.

A striking pattern emerges: supervisor rerank dominates all four
$\varepsilon = 0.75$ cells on both fairness and worst-group criteria. At
lower corruption, the winners are far more varied, with self-consistency,
EMA, trimmed vote, and majority all appearing in different cells. The margins
between the best and runner-up are typically very small (often $< 0.001$),
reinforcing our main finding that non-oracle methods are largely
interchangeable in benign conditions.

For worst-group utility, the pattern is similar but with one notable
difference: trimmed vote appears more frequently as a winner at moderate
corruption ($\varepsilon = 0.25$--$0.5$), consistent with its
breakdown-point property of discarding the most harmful agents. MW wins
the deceptive $\varepsilon = 0.5$ cell for worst-group utility (0.308),
suggesting that its weight tracking is particularly effective at protecting
minority classes against persuasive adversaries.

\begin{table}[H]
\centering
\scriptsize
\caption{Fairness regime winners (Jain's index, FH). Best non-oracle
  method per regime. Margin is the gap to the runner-up.}
\label{tab:fairness_regime}
\begin{tabular}{@{}llllr@{}}
\toprule
Adversary & $\varepsilon$ & Best & Jain & Margin \\
\midrule
Coordinated & 0.00 & Self-Cons  & .940 & .0002 \\
Coordinated & 0.75 & Supervisor & .935 & .0027 \\
Deceptive   & 0.00 & Conf-Wt    & .937 & .0000 \\
Deceptive   & 0.75 & Supervisor & .939 & .0015 \\
Scheduled   & 0.50 & EMA        & .943 & .0000 \\
Scheduled   & 0.75 & Supervisor & .934 & .0036 \\
Selfish     & 0.50 & Supervisor & .938 & .0013 \\
Selfish     & 0.75 & Trimmed    & .940 & .0005 \\
\bottomrule
\end{tabular}
\end{table}

\begin{table}[H]
\centering
\scriptsize
\caption{Worst-group utility regime winners (FH). Best non-oracle method
  per regime.}
\label{tab:wg_regime}
\begin{tabular}{@{}llllr@{}}
\toprule
Adversary & $\varepsilon$ & Best & Worst-Grp & Margin \\
\midrule
Coordinated & 0.00 & Self-Cons  & .315 & .0008 \\
Coordinated & 0.75 & Supervisor & .305 & .0121 \\
Deceptive   & 0.50 & MW         & .308 & .0005 \\
Deceptive   & 0.75 & Supervisor & .312 & .0115 \\
Scheduled   & 0.50 & Trimmed    & .318 & .0002 \\
Scheduled   & 0.75 & Supervisor & .300 & .0072 \\
Selfish     & 0.25 & Conf-Wt    & .299 & .0000 \\
Selfish     & 0.75 & Trimmed    & .316 & .0010 \\
\bottomrule
\end{tabular}
\end{table}

\section{Full Pareto Grid}
\label{sec:pareto_full}

Tables~\ref{tab:pareto_util} and \ref{tab:pareto_fair} present the complete
$5 \times 5$ Pareto grid in matrix form, with welfare weight $\alpha$ as
rows and fairness weight $\beta$ as columns. The $\alpha = 1.0$ row
consistently dominates. The Pareto-optimal point $(\alpha = 1.0,
\beta = 0.25)$ achieves the highest utility (0.479) and fairness (0.939)
simultaneously. Performance degrades substantially when $\alpha < 0.5$,
indicating that welfare-oriented loss is essential for maintaining utility.

\begin{table}[H]
\centering
\scriptsize
\caption{MW Pareto grid: utility. Rows are $\alpha$, columns are $\beta$.
  Bold marks the Pareto-optimal point.}
\label{tab:pareto_util}
\begin{tabular}{@{}c|ccccc@{}}
\toprule
 & \multicolumn{5}{c}{$\beta$} \\
$\alpha$ & 0 & 0.25 & 0.50 & 0.75 & 1.00 \\
\midrule
0.00 & .432 & .414 & .425 & .430 & .431 \\
0.25 & .426 & .421 & .424 & .416 & .422 \\
0.50 & .403 & .416 & .407 & .422 & .406 \\
0.75 & .417 & .414 & .440 & .413 & .423 \\
1.00 & .423 & \textbf{.479} & .451 & .455 & .462 \\
\bottomrule
\end{tabular}
\end{table}

\begin{table}[H]
\centering
\scriptsize
\caption{MW Pareto grid: Jain's fairness index. Same layout as
  Table~\ref{tab:pareto_util}.}
\label{tab:pareto_fair}
\begin{tabular}{@{}c|ccccc@{}}
\toprule
 & \multicolumn{5}{c}{$\beta$} \\
$\alpha$ & 0 & 0.25 & 0.50 & 0.75 & 1.00 \\
\midrule
0.00 & .928 & .922 & .929 & .924 & .935 \\
0.25 & .920 & .918 & .921 & .925 & .920 \\
0.50 & .916 & .913 & .923 & .928 & .922 \\
0.75 & .917 & .918 & .917 & .925 & .916 \\
1.00 & .923 & \textbf{.939} & .927 & .928 & .935 \\
\bottomrule
\end{tabular}
\end{table}

% ===========================================================================
\bibliographystyle{unsrtnat}
\begin{thebibliography}{12}

\bibitem[Arora et~al.(2012)]{arora2012multiplicative}
S.~Arora, E.~Hazan, and S.~Kale.
\newblock The multiplicative weights update method: a meta-algorithm and
  applications.
\newblock {\em Theory of Computing}, 8(1):121--164, 2012.

\bibitem[Blanchard et~al.(2017)]{blanchard2017machine}
P.~Blanchard, E.~M.~El~Mhamdi, R.~Guerraoui, and J.~Stainer.
\newblock Machine learning with adversaries: Byzantine tolerant gradient
  descent.
\newblock In {\em NeurIPS}, 2017.

\bibitem[Chen et~al.(2023)]{chen2023frugalgpt}
L.~Chen, M.~Zaharia, and J.~Zou.
\newblock FrugalGPT: How to use large language models while reducing cost and
  improving performance.
\newblock {\em arXiv preprint arXiv:2305.05176}, 2023.

\bibitem[Du et~al.(2023)]{du2023debate}
Y.~Du, S.~Li, A.~Torralba, J.~B.~Tenenbaum, and I.~Mordatch.
\newblock Improving factuality and reasoning in language models through
  multiagent debate.
\newblock {\em arXiv preprint arXiv:2305.14325}, 2023.

\bibitem[Jiang et~al.(2023)]{jiang2023llmblender}
D.~Jiang, X.~Ren, and B.~Y.~Lin.
\newblock LLM-Blender: Ensembling large language models with pairwise ranking
  and generative fusion.
\newblock In {\em ACL}, 2023.

\bibitem[Li et~al.(2024)]{li2024more}
Z.~Li et~al.
\newblock More agents is all you need.
\newblock {\em arXiv preprint arXiv:2402.05120}, 2024.

\bibitem[Liang et~al.(2023)]{liang2023encouraging}
T.~Liang, Z.~He, W.~Jiao, X.~Wang, Y.~Wang, R.~Wang, Y.~Yang, Z.~Tu, and
  S.~Shi.
\newblock Encouraging divergent thinking in large language models through
  multi-agent debate.
\newblock {\em arXiv preprint arXiv:2305.19118}, 2023.

\bibitem[Lugosi and Mendelson(2019)]{lugosi2019mean}
G.~Lugosi and S.~Mendelson.
\newblock Mean estimation and regression under heavy-tailed distributions: A
  survey.
\newblock {\em Foundations of Computational Mathematics}, 19(5):1145--1190,
  2019.

\bibitem[Mazeika et~al.(2024)]{mazeika2024harmbench}
M.~Mazeika et~al.
\newblock HarmBench: A standardized evaluation framework for automated red
  teaming and robust refusal.
\newblock {\em arXiv preprint arXiv:2402.04249}, 2024.

\bibitem[Wang et~al.(2023)]{wang2023selfconsistency}
X.~Wang, J.~Wei, D.~Schuurmans, Q.~Le, E.~Chi, S.~Narang, A.~Chowdhery, and
  D.~Zhou.
\newblock Self-consistency improves chain of thought reasoning in language
  models.
\newblock In {\em ICLR}, 2023.

\bibitem[Zou et~al.(2023)]{zou2023universal}
A.~Zou, Z.~Wang, J.~Z.~Kolter, and M.~Fredrikson.
\newblock Universal and transferable adversarial attacks on aligned language
  models.
\newblock {\em arXiv preprint arXiv:2307.15043}, 2023.

\end{thebibliography}

\end{document}
